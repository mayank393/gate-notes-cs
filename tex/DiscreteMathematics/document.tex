\documentclass{book}

\begin{document}
\chapter {Discrete Mathematics}

\section {Propositional and First Order Logic}
\subsection {Propositions}
A \textbf{Proposition} is a declarative sentence (that is, a sentence that declares a fact) that is either true or false, but not both. It is the basic building block of logic.

examples.
\begin{center}
\begin{tabular}{ |c|c|c|c|}
\hline
Statement & Proposition & Truth Value  \\ \hline
Delhi is the capital of India. & Yes & True \\
Pluto is a planet. & Yes & False \\
1 + 1 = 2 & Yes & True  \\
2 + 5 = 0 & Yes & False  \\
What time is it? & No & N/A \\
Read this carefully. & No & N/A \\
x  + 1 = 2 & No & N/A \\ \hline
\end{tabular}
\end{center}

\textbf{Note:} Proposition must have a truth value. For x + 1 = 2, The truth value cannot be determined without the value for x. These statements can be converted in to propositions using Quantifiers.

\textbf{Propositional variables}(or \textbf{statement variables}) are variables used to represent propositions. Conventional letters used for proportional variables are p, q, r, s, ... .

The \textbf{truth value} of a proposition is true, denoted by T, if it is a true proposition, and the truth value of a proposition is false, denoted by F, it it is a false proposition.

\textbf{propositional calculus} or \textbf{propositional logic} is the area of logic that deals with propositions.

Many mathematical statements are constructed by combining one or more propositions. New propositions, called \textbf{compound propositions} are formed from existing propositions using logical operators.

\subsection {Logical Operators}
\subsection {Quantifiers}
\subsection {Identities}
\subsection {Fallacies}

\section {Set Theory}
\subsection{Definition}
\subsection{Naive Set Theory}
\subsection{Notations}
\subsection{Set Operators}

\section{Relations}
\subsection{Functions}
\subsection{Partial Order}
\subsection{Lattice}

\section{Groups}
\subsection{Group}
\subsection{Monoid}

\section {Graphs}
\subsection {Connectivity}
\subsection {Matching}
\subsection {Coloring}

\section {Combinatorics}
\subsection {Counting}
\subsection {Recurrence Relations}
\subsection {Generating Functions}
\end{document}