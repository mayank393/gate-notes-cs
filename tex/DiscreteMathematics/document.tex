\documentclass{book}

\usepackage{multicol}

\begin{document}
\chapter {Discrete Mathematics}

\section {Propositional and First Order Logic}
\subsection {Propositions}
A \textbf{Proposition} is a declarative sentence (that is, a sentence that declares a fact) that is either true or false, but not both. It is the basic building block of logic.

examples.
\begin{center}
\begin{tabular}{ |c|c|c|c|}
\hline
Statement & Proposition & Truth Value  \\ \hline
Delhi is the capital of India. & Yes & True \\
Pluto is a planet. & Yes & False \\
1 + 1 = 2 & Yes & True  \\
2 + 5 = 0 & Yes & False  \\
What time is it? & No & N/A \\
Read this carefully. & No & N/A \\
x  + 1 = 2 & No & N/A \\ \hline
\end{tabular}
\end{center}

\textbf{Note:} Proposition must have a truth value. For x + 1 = 2, The truth value cannot be determined without the value for x. These statements can be converted in to propositions using Quantifiers.

\textbf{Propositional variables}(or \textbf{statement variables}) are variables used to represent propositions. Conventional letters used for proportional variables are p, q, r, s, ... .

The \textbf{truth value} of a proposition is true, denoted by T, if it is a true proposition, and the truth value of a proposition is false, denoted by F, it it is a false proposition.

\textbf{propositional calculus} or \textbf{propositional logic} is the area of logic that deals with propositions.

Many mathematical statements are constructed by combining one or more propositions. New propositions, called \textbf{compound propositions} are formed from existing propositions using logical operators.

\subsection {Logical Operators}
\begin{enumerate}
\item \textbf{Negation}
Let \emph{p} be a proposition. The \emph{negation of p}, denoted by $\lnot p$
(also denoted by $\overline p$), is the statement
\begin{center}
"It is not the case that p"
\end{center}
The proposition $\lnot p$ is read "not p". The truth value of the negation of p, i.e. $\lnot p$, is the opposite of the truth value of p.

\begin{center}
\begin{tabular}{ |c|c|}
\hline
$p$ & $\neg p$  \\ \hline
False & True \\ 
True & False \\ \hline
\end{tabular}
\end{center}

\item \textbf{Conjunction}
Let \emph{p} and \emph{q} be propositions. The \emph{conjunction} of \emph{p} and \emph{q}, denoted by $p \land q$ is true when both \emph{p} and \emph{q} are true and is false otherwise.

\begin{center}
\begin{tabular}{ |c|c|c|}
\hline
$p$ & $q$ & $p \land q$  \\ \hline
False & False & False \\
False & True & False \\
True & False & False \\
True & True & True \\ \hline
\end{tabular}
\end{center}

\item \textbf{Disjunction}
Let \emph{p} and \emph{q} be propositions. The \emph{disjunction} of \emph{p} and \emph{q}, denoted by $p \lor q$ is false when both \emph{p} and \emph{q} are false and is true otherwise.

\begin{center}
\begin{tabular}{ |c|c|c|}
\hline
$p$ & $q$ & $p \lor q$  \\ \hline
False & False & False \\
False & True & True \\
True & False & True \\
True & True & True \\ \hline
\end{tabular}
\end{center}

\item \textbf{Exclusive Or}
Let \emph{p} and \emph{q} be propositions. The \emph{exclusive or} of \emph{p} and \emph{q}, denoted by $p \oplus q$, is the proposition that is true when exactly one of \emph{p} and \emph{q} is true and is false otherwise.

\begin{center}
\begin{tabular}{ |c|c|c|}
\hline
$p$ & $q$ & $p \oplus q$  \\ \hline
False & False & False \\
False & True & True \\
True & False & True \\
True & True & False \\ \hline
\end{tabular}
\end{center}

\item \textbf{Conditional}
Let \emph{p} and \emph{q} be propositions. The \emph{Conditional statement} $p \to q$ is the proposition "if p, then q". The conditional statement $p \to q$ is false when \emph{p} is true and \emph{q} is false, and true otherwise. In the conditional statment $p \to q$, \emph{p} is called the hypothesis(or \emph{antecedent} or \emph{premise}) and \emph{q} is called the \emph{conclusion}(or \emph{consequence}).

\begin{center}
\begin{tabular}{ |c|c|c|}
\hline
$p$ & $q$ & $p \to q$  \\ \hline
False & False & True \\
False & True & True \\
True & False & False \\
True & True & True \\ \hline
\end{tabular}
\end{center}

other ways to describe \emph{Conditional Statements}:
\begin{multicols}{2}
\begin{enumerate}
\item "if \emph{p}, then \emph{q}"
\item "if \emph{p}, \emph{q}"
\item "\emph{p} is sufficient for \emph{q}"
\item "\emph{q} if \emph{p}"
\item "\emph{q} when \emph{p}"
\item "a neccessary condition for \emph{p} is \emph{q}"
\item "\emph{q} unless $\lnot p$"
\item "\emph{p} implies \emph{q}"
\item "\emph{p} only if \emph{q}"
\item "a sufficient condition for \emph{q} is \emph{p}"
\item "\emph{q} whenever \emph{p}"
\item "\emph{q} is necessary for \emph{p}"
\item "\emph{q} follows from \emph{p}"
\end{enumerate}
\end{multicols}

\textbf{CONVERSE, CONTRAPOSITIVE AND INVERSE} We can form some new conditional statements starting with a conditional statement $p \to q$.
\begin{enumerate}
\item The proposition $q \to p$ is called the \emph{Converse} of $p \to q$.
\item The proposition $\lnot q \to \lnot p$ is called the \emph{Contrapositive} of $p \to q$.
\item The proposition $\lnot p \to \lnot q$ is called the \emph{Inverse} of $p \to q$.
\end{enumerate}
\textbf{Note:} 
\begin{enumerate}
\item When two compound propositions always have the same truth value we call them \textbf{equivalent}.
\item Conditional ($p \to q$) and Contrapositive($\lnot q \to \lnot p$) are equivalent.
\item Converse ($q \to p$) and Inverse($\lnot p \to \lnot q$) are equivalent.
\end{enumerate}

\item \textbf{Biconditional}
Let \emph{p} and \emph{q} be propositions. The \emph{biconditional statement} $p \leftrightarrow q$ is a propostion "\emph{p} if and only if \emph{q}".  The biconditional statement $p \leftrightarrow q$ is true when \emph{p} and \emph{q} have the same truth values, and is false otherwise. Biconditional statements are also called \emph{bi-implications}.

\begin{center}
\begin{tabular}{ |c|c|c|}
\hline
$p$ & $q$ & $p \leftrightarrow q$  \\ \hline
False & False & True \\
False & True & False \\
True & False & False \\
True & True & True \\ \hline
\end{tabular}
\end{center}

other ways to describe \emph{Biconditional Statements}:
\begin{enumerate}
\item "\emph{p} is necessary and sufficient for \emph{q}"
\item "if \emph{p} then \emph{q}, and conversely"
\item "\emph{p} iff \emph{q}"
\end{enumerate}

\textbf{Note:} 
\begin{enumerate}
\item $p \leftrightarrow q$ is equivalent to $(p \to q) \land (q \to p)$
\item Biconditionals are not always explicit in natural language. In particular, the "if and only if" consturction used in biconditionals is rarely used in common language. Instead, biconditionals are often expressed using "if, then" or on "only if" construction. The other part of the  "if and only if" is implicit.

ex. "If you finish your meal, then you can have dessert." What is really ment is "You can have dessert if and only if you finish your meal"
\end{enumerate}
\end{enumerate}

\subsection {Quantifiers}
\subsection {Identities}
\subsection {Fallacies}

\section {Set Theory}
\subsection{Definition}
\subsection{Naive Set Theory}
\subsection{Notations}
\subsection{Set Operators}

\section{Relations}
\subsection{Functions}
\subsection{Partial Order}
\subsection{Lattice}

\section{Groups}
\subsection{Group}
\subsection{Monoid}

\section {Graphs}
\subsection {Connectivity}
\subsection {Matching}
\subsection {Coloring}

\section {Combinatorics}
\subsection {Counting}
\subsection {Recurrence Relations}
\subsection {Generating Functions}
\end{document}